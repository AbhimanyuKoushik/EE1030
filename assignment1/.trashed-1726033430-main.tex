%iffalse
\let\negmedspace\undefined
\let\negthickspace\undefined
\documentclass[journal,12pt,twocolumn]{IEEEtran}
\usepackage{cite}
\usepackage{amsmath,amssymb,amsfonts,amsthm}
\usepackage{algorithmic}
\usepackage{graphicx}
\usepackage{textcomp}
\usepackage{xcolor}
\usepackage{txfonts}
\usepackage{listings}
\usepackage{enumitem}
\usepackage{mathtools}
\usepackage{gensymb}
\usepackage{comment}
\usepackage[breaklinks=true]{hyperref}
\usepackage{tkz-euclide} 
\usepackage{listings}
\usepackage{gvv}                                        
%\def\inputGnumericTable{}                                 
\usepackage[latin1]{inputenc}     
\usepackage{xparse}
\usepackage{color}                                            
\usepackage{array}                                            
\usepackage{longtable}                                       
\usepackage{calc}                                             
\usepackage{multirow}                                         
\usepackage{hhline}                                           
\usepackage{ifthen}                                           
\usepackage{lscape}
\usepackage{tabularx}
\usepackage{array}
\usepackage{float}
\newtheorem{theorem}{Theorem}[section]
\newtheorem{problem}{Problem}
\newtheorem{proposition}{Proposition}[section]
\newtheorem{lemma}{Lemma}[section]
\newtheorem{corollary}[theorem]{Corollary}
\newtheorem{example}{Example}[section]
\newtheorem{definition}[problem]{Definition}
\newcommand{\BEQA}{\begin{eqnarray}}
\newcommand{\EEQA}{\end{eqnarray}}
\newcommand{\define}{\stackrel{\triangle}{=}}
\theoremstyle{remark}
\newtheorem{rem}{Remark}
% Marks the beginning of the document
\begin{document}
\title{Assignment 1}
\author{EE24Btech11024 - G. Abhimanyu Koushik}
\maketitle
\renewcommand{\thefigure}{\theenumi}
\renewcommand{\thetable}{\theenumi}
\subsection*{D: Single Correct}
\begin{enumerate}
\item Circle(s) touching the x-axis at a distance \brak{3} from the origin and having an intercept of length $2\sqrt{7}$ on the y-axis is \brak{\text{are}}

\hfill{\brak{JEE Adv. 2013}}
\begin{enumerate}[label=(\alph*)]
\item $x^2 + y^2 - 6x + 8y + 9 = 0$
\item $x^2 + y^2 - 6x + 7y + 9 = 0$
\item $x^2 + y^2 - 6x - 8y + 9 = 0$
\item $x^2 + y^2 - 6x - 7y + 9 = 0$
\end{enumerate}
\item A circle $\textit{S}$ passes through the point \brak{0,1} and is orthogonal to the circle $(x-1)^2+y^2=16$ and $x^2+y^2=1$. Then

\hfill {\brak{JEE Adv. 2014}}
\begin{enumerate}[label=(\alph*)]
\item Radius of $\textit{S}$ is $8$
\item Radius of $\textit{S}$ is $7$
\item Centre of $\textit{S}$ is \brak{-7,1}
\item Centre of $\textit{S}$ is \brak{-8,1}
\end{enumerate}
\item Let $\textit{RS}$ be the diameter of the Circle $x^{2} + y^{2} = 1$, where $\textit{S}$ is the point \brak{1,0}. Let $\textit{P}$ be a variable point \brak{\text{other than R and S}} on the circle and tangents to the circle at $\textit{S}$ and $\textit{P}$ meet at the point $\textit{Q}$. The normal to the circle at $\textit{P}$ intersects a line drawn through $\textit{Q}$ parallel to $\textit{RS}$ at point $\textit{E}$. Then the locus of $\textit{E}$ passes through the point \brak{s}

\hfill {\brak{JEE Adv. 2016}}
\begin{enumerate}[label=(\alph*)]
\item $\left(\frac{1}{3}, \frac{1}{\sqrt{3}}\right)$
\item $\left(\frac{1}{4}, \frac{1}{2} \right)$
\item $\left(\frac{1}{3}, -\frac{1}{\sqrt{3}} \right)$
\item $\left(\frac{1}{4}, -\frac{1}{2}\right)$
\end{enumerate}
\item Let \textit{$T$} be a line passing through the points \textit{$P$}\brak{-2,7} and \textit{$Q$}\brak{2,-5}. Let \textit{$F_1$} be the set of all pairs of circles \brak{\textit{$S_1$},\textit{$S_2$}} such that \textit{$T$} is tangent to \textit{$S_1$} at \textit{$P$} and tangent to \textit{$S_2$} at \textit{$Q$}, and also such that \textit{$S_1$} and \textit{$S_2$} touch each other at a point, say \textit{$M$}. Let \textit{$E_1$} be the set representing the locus of \textit{$M$} as the pair \brak{\textit{$S_1$},\textit{$S_2$}} varies in \textit{$F_1$}. Let the set of all straight line segments joining a pair of distinct points of \textit{$E_1$} and passing through the point \textit{$R$}\brak{1,1} be \textit{$F_2$}. Then which of the following statements is (are) TRUE?

\hfill{\brak{JEE Adv. 2018}}
\begin{enumerate}
\item The point \brak{-2,7} lies on \textit{$E_1$}
\item The point \brak{\frac{4}{5}, \frac{7}{5}} does \textbf{NOT} lie on \textit{$E_1$}
\item The point \brak{\frac{1}{3},1} lies on \textit{$E_1$}
\item The point \brak{0, \frac{3}{2}} does not lie on \textit{$E_1$}
\end{enumerate}
\end{enumerate}
\subsection*{E: Subjective}
\begin{enumerate}
\item Find the equation of the circle whose radius is 5 and which touches the circle $x^2+y^2-2x-4y-20=0$ at the point \brak{5,5}

\hfill {\brak{1978}}
\item Let $\textit{A}$ be the centre of circle $x^2+y^2-2x-4y-20=0$. Suppose that the tangents at the points $\textit{B}$\brak{1,7} and $\textit{CD}$ \brak{4,-2} on the circle meet at point $\textit{C}$. Find the area of the quadrilateral $ABCD$.

\hfill {\brak{1981 - 4 marks}}
\item Find the equations of the circle passing through \brak{-4,3} and touching the lines $x+y=2$ and $x-y=2$

\hfill {\brak{1981 - 4 marks}}
\item Through a fixed point \brak{h,k} secants are drawn to the circle $x^2+y^2=r^2$. Show that the locus of the mid-points of the secants intercepted is $x^2+y^2=hx+ky$

\hfill {\brak{1983 - 5 marks}}
\item The abscissa of two points $\textit{A}$ and $\textit{B}$ are roots of the equation $x^2+2ax-b^2=0$ and their ordinates are roots of the equation $x^2+2px-q^2=0$. Find the equation and the radius of the circle with $\textit{AB}$ as diameter.

\hfill {\brak{1984 - 4 marks}}
\item Lines $5x+12y-10=0$ and $5x-12y-40=0$ touch a Circle \textit{$C_1$} of diameter 6. If the centre of \textit{$C_1$} lies in the first quadrant, find the equation of circle \textit{$C_2$} which is concentric with \textit{$C_1$} and cuts intecepts of length 8 on these lines

\hfill {\brak{1986 - 5 marks}}
\item Let a given Line \textit{$L_1$} intersects the $x$ and $y$ axes at \textit{$P$} and \textit{$Q$} respectively. Let another line \textit{$L_2$}, perpendicular to \textit{$L_1$}, cut the $x$ and $y$ axes at \textit{$R$} and \textit{$S$}, respectively. Show that the locus of the point of intersection of \textit{$PS$} and \textit{$QR$} is a circle passing through origin.

\hfill {\brak{1987 - 3 marks}}
\item The circle $x^2+y^2-4x-y+4=0$ is inscribed in a triangle which has two of its sides along the co-ordinate axes. The locus of circumcentre of the triangle is $x+y-xy+k(x^2+y^2)\textsuperscript{1/2}$. Find $\textit{k}$.

\hfill {\brak{1987 - 4 marks}}
\item If $\left( m_i, \frac{1}{m_i} \right), m_i > 0, i = 1, 2, 3, 4$ are four distinct points on a circle, then show that $m_1m_2m_3m_4=1$

\hfill {\brak{1989 - 2 marks}}
\item A circle touches the line $y=x$ at a point $\textit{P}$ such that $OP=4\sqrt{2}$ , where O is the origin. The circle contains the point \brak{-10,2} in its interior and the length of its chord on the line $x+y=0$ is $6\sqrt{2}$. Determine the equation of circle.

\hfill {\brak{1990 - 5 marks}}
\item Two circles, each of radius 5 units, touch each other at \brak{1,2}. If the equation of common tangent is $4x+3y=10$, find the equations of circles.

\hfill {\brak{1991 - 4 marks}}
\end{enumerate}
\end{document}

