%iffalse
\let\negmedspace\undefined
\let\negthickspace\undefined
\documentclass[journal,12pt,onecolumn]{IEEEtran}
\usepackage{cite}
\usepackage{amsmath,amssymb,amsfonts,amsthm}
\usepackage{algorithmic}
\usepackage{graphicx}
\usepackage{textcomp}
\usepackage{xcolor}
\usepackage{txfonts}
\usepackage{listings}
\usepackage{enumitem}
\usepackage{mathtools}
\usepackage{gensymb}
\usepackage{comment}
\usepackage[breaklinks=true]{hyperref}
\usepackage{tkz-euclide} 
\usepackage{listings}
\usepackage{gvv}                                        
%\def\inputGnumericTable{}                                 
\usepackage[latin1]{inputenc}     
\usepackage{xparse}
\usepackage{color}                                            
\usepackage{array}                                            
\usepackage{longtable}                                       
\usepackage{calc}                                             
\usepackage{multirow}
\usepackage{multicol}
\usepackage{hhline}                                           
\usepackage{ifthen}                                           
\usepackage{lscape}
\usepackage{tabularx}
\usepackage{array}
\usepackage{float}
\newtheorem{theorem}{Theorem}[section]
\newtheorem{problem}{Problem}
\newtheorem{proposition}{Proposition}[section]
\newtheorem{lemma}{Lemma}[section]
\newtheorem{corollary}[theorem]{Corollary}
\newtheorem{example}{Example}[section]
\newtheorem{definition}[problem]{Definition}
\newcommand{\BEQA}{\begin{eqnarray}}
\newcommand{\EEQA}{\end{eqnarray}}
\newcommand{\define}{\stackrel{\triangle}{=}}
\theoremstyle{remark}
\newtheorem{rem}{Remark}
% Marks the beginning of the document
\begin{document}
\title{Assignment 1}
\author{EE24Btech11024 - G. Abhimanyu Koushik}
\maketitle
\renewcommand{\thefigure}{\theenumi}
\renewcommand{\thetable}{\theenumi}
\begin{enumerate}
\item For which of the following ordered pairs \brak{\mu,\delta} the system of linear equations\newline
$x+2y+3z=1$\newline
$3x+4y+5z=\mu\newline
4x+4y+4z=\delta$\newline
is inconsistent?
\begin{enumerate}
\begin{multicols}{4}
\item \brak{4,6}
\item \brak{3,4}
\item \brak{1,0}
\item \brak{4,3}
\end{multicols}
\end{enumerate}

\item Let $y=f\brak{x}$ be a solution to the differential equation\newline
$\sqrt{1-x^2}\frac{dy}{dx}+\sqrt{1-y^2} = 0 \text{, } \abs{x}<1$\newline
If $y\brak{\frac{1}{2}} = \frac{\sqrt{3}}{2}$, then $y\brak{-\frac{1}{\sqrt{2}}}$ is equal to
\begin{enumerate}
\begin{multicols}{4}
\item $-\frac{1}{\sqrt{2}}$
\item $-\frac{\sqrt{3}}{2}$
\item $\frac{1}{\sqrt{2}}$
\item $\frac{\sqrt{3}}{2}$
\end{multicols}
\end{enumerate}

\item If $a$, $b$ and $c$ are the greatest values of $\comb{19}{p}$, $\comb{20}{q}$, $\comb{21}{r}$ respectively, then:
\begin{enumerate}
\begin{multicols}{4}
\item $\brak{\frac{a}{11}} = \brak{\frac{b}{22}} = \brak{\frac{c}{42}}$
\item $\brak{\frac{a}{10}} = \brak{\frac{b}{11}} = \brak{\frac{c}{42}}$
\item $\brak{\frac{a}{11}} = \brak{\frac{b}{22}} = \brak{\frac{c}{21}}$
\item $\brak{\frac{a}{10}} = \brak{\frac{b}{11}} = \brak{\frac{c}{21}}$
\end{multicols}
\end{enumerate}

\item Which of the following is a tautology?
\begin{enumerate}
\begin{multicols}{4}
\item $\brak{P\wedge\brak{P\rightarrow Q}}\rightarrow Q$
\item $P\wedge \brak{P\vee Q}$
\item $\brak{Q\rightarrow\brak{\wedge\brak{P\rightarrow Q}}}$
\item $P\vee \brak{P\wedge Q}$
\end{multicols}
\end{enumerate}

\item Let $f:\mathbb{R}\rightarrow\mathbb{R}$ be such that for all $x\in\mathbb{R}$, \brak{2^{1+x}+2^{1-x}}, $f\brak{x}$ and \brak{3^x+3^{-x}} are in A.P, then the minimum value of $f\brak{x}$ is:
\begin{enumerate}
\begin{multicols}{4}
\item $0$
\item $4$
\item $3$
\item $2$
\end{multicols}
\end{enumerate}

\item The locus of a point which divides the line segment joining the point \brak{0,-1} and a point on parabola, $x^2 = 4y$, internally in the ration $1:2$ is:
\begin{enumerate}
\begin{multicols}{4}
\item $9x^2-12y=8$
\item $4x^2-3y=2$
\item $x^2-3y=2$
\item $9x^2-3y=2$
\end{multicols}
\end{enumerate}

\item For $a>0$, let the curves $C_1$:$y^2=ax$ and $C_2$:$x^2=ay$ intersect at origin $\vec{O}$ and a point $\vec{P}$. Let the line $x=b\brak{0<b<a}$ intersect the chord $OP$ and the x-axis at points $\vec{Q}$ and $\vec{R}$, respectively. If the line $x=b$ bisects the area bounded by the curves, $C_1$ and $C_2$, and the area of $\Delta OQR = \frac{1}{2}$, then $a$ satisfies the equation
\begin{enumerate}
\begin{multicols}{4}
\item $x^6-12x^3+4=0$
\item $x^6-12x^3-4=0$
\item $x^6+6x^3-4=0$
\item $x^6-6x^3+4=0$
\end{multicols}
\end{enumerate}

\item The inverse of the function $f\brak{x}=\frac{8^{2x}-8^{-2x}}{8^{2x}+8^{-2x}}$ is
\begin{enumerate}
\begin{multicols}{4}
\item $\frac{1}{4}\brak{\log_8{e}}\log_e{\brak{\frac{1+x}{1-x}}}$
\item $\frac{1}{4}\brak{\log_8{e}}\log_e{\brak{\frac{1-x}{1+x}}}$
\item $\frac{1}{4}\log_e{\brak{\frac{1+x}{1-x}}}$
\item $\frac{1}{4}\log_e{\brak{\frac{1-x}{1+x}}}$
\end{multicols}
\end{enumerate}

\item $\lim_{x\to 0}\brak{\frac{3x^2+2}{7x^2+2}}^{\frac{1}{x^2}}$ is equal to
\begin{enumerate}
\begin{multicols}{4}
\item $e$
\item $\frac{1}{e^2}$
\item $\frac{1}{e}$
\item $e^2$
\end{multicols}
\end{enumerate}

\item Let $f\brak{x}=\brak{\sin\brak{{\tan^{-1}x}}+\sin\brak{{\cot^{-1}x}}}^2-1$ where $\abs{x}>1$. If\newline
$\frac{dy}{dx}=\frac{1}{2}\frac{d}{dx}\brak{\sin^{-1}f\brak{x}}$\newline
and $y\brak{\sqrt{3}}=\frac{\pi}{6}$, then $y\brak{-\sqrt{3}}$ is equal to:
\begin{enumerate}
\begin{multicols}{4}
\item $\frac{\pi}{3}$
\item $\frac{2\pi}{3}$
\item $-\frac{\pi}{6}$
\item $\frac{5\pi}{6}$
\end{multicols}
\end{enumerate}

\item If the equation, $x^2+bx+45=0 \brak{b\in\mathbb{R}}$ has conjugate complex roots and they satisfy $\abs{z+1}=2\sqrt{10}$, then:
\begin{enumerate}
\begin{multicols}{4}
\item $b^2+b=12$
\item $b^2-b=42$
\item $b^2-b=30$
\item $b^2+b=72$
\end{multicols}
\end{enumerate}

\item The mean and standard deviation of $10$ observations are $20$ and $2$ respectively. Each of these $10$ observations is multiplied by $p$ and then reduced by $q$, where $p\neq 0$ and $q\neq 0$. If the new mean and standard deviation become half of their original values, then $q$ is equal to:
\begin{enumerate}
\begin{multicols}{4}
\item $-20$
\item $-5$
\item $10$
\item $-10$
\end{multicols}
\end{enumerate}

\item If $\int \frac{\cos x}{\sin^{3}x\brak{1+\sin^{6}x}^{\frac{2}{3}}}dx = f\brak{x}\brak{1+\sin^{6}x}^{\frac{1}{\lambda}}+c$, where $c$ is a constant of integration, then $\lambda f\brak{\frac{\pi}{3}}$ is equal to:
\begin{enumerate}
\begin{multicols}{4}
\item $-\frac{9}{8}$
\item $\frac{9}{8}$
\item $2$
\item $-2$
\end{multicols}
\end{enumerate}

\item Let $A$ and $B$ be two independent events such that $P\brak{A}=\frac{1}{3}$ and $P\brak{B}=\frac{1}{6}$. Then which of the following is \textbf{TRUE}?
\begin{enumerate}
\begin{multicols}{4}
\item $P\brak{\frac{A}{A\cup B}}=\frac{1}{4}$
\item $P\brak{\frac{A}{B'}}=\frac{1}{3}$
\item $P\brak{\frac{A}{B}}=\frac{2}{3}$
\item $P\brak{\frac{A'}{B'}}=\frac{1}{3}$
\end{multicols}
\end{enumerate}

\item If volume of a parallelepiped whose coterminous edges are given by\newline
$\overrightarrow{u} = \hat{i}+\hat{j}+\lambda\hat{k}$,\newline
$\overrightarrow{v} = \hat{i}+\hat{j}+3\hat{k}$ and\newline
$\overrightarrow{w} = 2\hat{i}+\hat{j}+\hat{k}$\newline
be $1$cu.unit. If $\theta$ is the angle between the edges $\overrightarrow{u}$ and $\overrightarrow{w}$ then, $\cos\theta$ will be:
\begin{enumerate}
\begin{multicols}{4}
\item $\frac{7}{6\sqrt{6}}$
\item $\frac{5}{7}$
\item $\frac{7}{6\sqrt{3}}$
\item $\frac{5}{3\sqrt{3}}$
\end{multicols}
\end{enumerate}

\end{enumerate}
\end{document}

