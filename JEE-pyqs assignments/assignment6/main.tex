%iffalse
\let\negmedspace\undefined
\let\negthickspace\undefined
\documentclass[journal,12pt,onecolumn]{IEEEtran}
\usepackage{cite}
\usepackage{amsmath,amssymb,amsfonts,amsthm}
\usepackage{algorithmic}
\usepackage{graphicx}
\usepackage{textcomp}
\usepackage{xcolor}
\usepackage{txfonts}
\usepackage{listings}
\usepackage{enumitem}
\usepackage{mathtools}
\usepackage{gensymb}
\usepackage{comment}
\usepackage[breaklinks=true]{hyperref}
\usepackage{tkz-euclide} 
\usepackage{listings}
\usepackage{gvv}                                        
%\def\inputGnumericTable{}                                 
\usepackage[latin1]{inputenc}     
\usepackage{xparse}
\usepackage{color}                                            
\usepackage{array}                                            
\usepackage{longtable}                                       
\usepackage{calc}                                             
\usepackage{multirow}
\usepackage{multicol}
\usepackage{hhline}                                           
\usepackage{ifthen}                                           
\usepackage{lscape}
\usepackage{tabularx}
\usepackage{array}
\usepackage{float}
\newtheorem{theorem}{Theorem}[section]
\newtheorem{problem}{Problem}
\newtheorem{proposition}{Proposition}[section]
\newtheorem{lemma}{Lemma}[section]
\newtheorem{corollary}[theorem]{Corollary}
\newtheorem{example}{Example}[section]
\newtheorem{definition}[problem]{Definition}
\newcommand{\BEQA}{\begin{eqnarray}}
\newcommand{\EEQA}{\end{eqnarray}}
\newcommand{\define}{\stackrel{\triangle}{=}}
\theoremstyle{remark}
\newtheorem{rem}{Remark}
% Marks the beginning of the document
\begin{document}
\title{Assignment 6}
\author{EE24Btech11024 - G. Abhimanyu Koushik}
\maketitle
\renewcommand{\thefigure}{\theenumi}
\renewcommand{\thetable}{\theenumi}
\subsection{Multiple Choice}
\begin{enumerate}

\item Let $9=x_1<x_2<\dots <x_7$ be in an A.P with common difference $d$. If the standard deviation of $x_1$, $x_2\dots$, $x_7$ is $4$ and the mean is $\bar{x}$, then $\bar{x}+x_6$ is equal to:

\hfill{\brak{\text{Jan 2023}}}
\begin{enumerate}
\begin{multicols}{4}
\item $18\brak{1+\frac{1}{\sqrt{3}}}$
\item $34$
\item $2\brak{9+\frac{8}{\sqrt{7}}}$
\item $25$
\end{multicols}
\end{enumerate}

\item For the system of linear equations $\alpha x+y+z=1$, $x+\alpha y+z=1$, $x+y+\alpha z=\beta$, which of the following is \textbf{NOT} correct?

\hfill{\brak{\text{Jan 2023}}}
\begin{enumerate}
\item It has infinitely many solutions if $\alpha=2$ and $\beta=-1$
\item It has no solution if $\alpha=-2$ and $\beta=1$
\item $x+y+z=\frac{3}{4}$ if $\alpha=2$ and $\beta=1$
\item It has infinitely many solutions if $\alpha=1$ and $\beta=1$
\end{enumerate}

\item Let $\vec{a}=5\vec{i}-\vec{j}-3\vec{k}$ and $\vec{b}=\vec{i}+3\vec{j}+5\vec{k}$ be two vectors. Then which one of the following statements is \textbf{TRUE}?

\hfill{\brak{\text{Jan 2023}}}
\begin{enumerate}
\item The projection of $\vec{a}$ on $\vec{b}$ is $-\frac{17}{\sqrt{35}}$ and the direction of the projection vector is opposite to the direction of $\vec{b}$	
\item The projection of $\vec{a}$ on $\vec{b}$ is $\frac{17}{\sqrt{35}}$ and the direction of the projection vector is opposite to the direction of $\vec{b}$
\item The projection of $\vec{a}$ on $\vec{b}$ is $\frac{17}{\sqrt{35}}$ and the direction of the projection vector is same as of $\vec{b}$ 
\item The projection of $\vec{a}$ on $\vec{b}$ is $-\frac{17}{\sqrt{35}}$ and the direction of the projection vector is same as of $\vec{b}$
\end{enumerate}

\item Let $P\brak{x_0,y_0}$ be the point on hyperbola $3x^2-4y^2=36$, which is nearest to the line $3x+2y=1$. Then $\sqrt{2}\brak{y_0-x_0}$ is equal to:
\hfill{\brak{\text{Jan 2023}}}
\begin{enumerate}
\begin{multicols}{4}
\item $-3$
\item $9$
\item $-9$
\item $3$
\end{multicols}
\end{enumerate}

\item If $y\brak{x}=x^x$, $x>0$, then ${y^\prime}^\prime\brak{2-2y^\prime\brak{2}}$ is equal to:

\hfill{\brak{\text{Jan 2023}}}
\begin{enumerate}
\begin{multicols}{4}
\item $8\log_e{2}-2$
\item $4\log_e{2}+2$
\item $4\brak{\log_e{2}}^2-2$
\item $4\brak{\log_e{2}}^2+2$
\end{multicols}
\end{enumerate}
\end{enumerate}

\subsection{Numericals}
\begin{enumerate}

\item The total number of six digit numbers formed using digits $4$, $6$, $9$ only and divisible by $6$ is \rule{1cm}{0.15mm}.

\hfill{\brak{\text{Jan 2023}}}

\item The number of integral solutions to the equation $x+y+z=21$, where $x\geq 1$, $y\geq 3$, $z\geq 4$ is equal to \rule{1cm}{0.15mm}.

\hfill{\brak{\text{Jan 2023}}}

\item The line $x=8$ is the directrix of the ellipse $E:\frac{x^2}{a^2}+\frac{y^2}{b^2}=1$ with the corresponding focus $\brak{2,0}$. If the tangent to $E$ at the point $\vec{P}$ in the first quadrant passes through the point $\brak{0,4\sqrt{3}}$ and intersects the x-axis at $\vec{Q}$, then $\brak{3PQ}^2$ is equal to \rule{1cm}{0.15mm}.

\hfill{\brak{\text{Jan 2023}}}

\item If the x-intercept of a focal chord of the parabola $y^2=8x+4y+4$ is 3, then the length of this chord is equal to \rule{1cm}{0.15mm}.

\hfill{\brak{\text{Jan 2023}}}

\item If $\int_{0}^{\pi}\frac{5^{\cos x}\brak{1+\cos x\cos 3x+\cos^{2}x+\cos^{3}x\cos 3x}}{1+5^{\cos x}}dx=\frac{k\pi}{16}$, then $k$ is equal to \rule{1cm}{0.15mm}.

\hfill{\brak{\text{Jan 2023}}}

\item Let the sixth term in the binomial expansion of $\brak{\sqrt{2^{\log_2{\brak{10-3^x}}}}+\sqrt[5]{2^{\brak{x-2}\log_2{3}}}}^m$, in the increasing powers of $2^{\brak{x-2}\log_2{3}}$, be $21$. If the binomial coefficients of second, third and fourth terms in the expansion are respectively the first, third and fifth terms of an A.P, then the sum of the squares of all possible values of $x$ is \rule{1cm}{0.15mm}.

\hfill{\brak{\text{Jan 2023}}}

\item If the term without $x$ in the expansion of $\brak{x^{\frac{2}{3}}+\frac{\alpha}{x^3}}^{22}$ is $7315$, then $\abs{\alpha}$ is equal to \rule{1cm}{0.15mm}.

\hfill{\brak{\text{Jan 2023}}}

\item The sum of the common terms of the following three arithmetic progressions.\newline
$3$, $7$, $11$, $15$, $\dots\dots\dots$, $399$\newline
$2$, $5$, $8$, $11$, $\dots\dots\dots$, $359$\newline
$2$, $7$, $12$, $17$, $\dots\dots\dots$, $197$ is equal to \rule{1cm}{0.15mm}. 

\hfill{\brak{\text{Jan 2023}}}

\item Let $\alpha x+\beta y+\gamma z=1$ be the equation of a plane passing through the point $\brak{3,-2,5}$ and perpendicular to line joining the points $\brak{1,2,3}$ and $\brak{-2,3,5}$. Then the value of $\alpha\beta\gamma$ is equal to \rule{1cm}{0.15mm}.

\hfill{\brak{\text{Jan 2023}}}

\item The point of intersection $\vec{C}$ of the plane $8x+y+2z=0$ and the line joining points $\vec{A}\brak{-3,-6,1}$ and $\vec{B}\brak{2,4,-3}$ divides the line segment $AB$ internally in the ratio $k:1$. If $a$, $b$, $c$ \brak{\abs{a}\text{, }\abs{b}\text{, }\abs{c}\text{ are coprimes}} are the direction ratios of the perpendicular from the point $\vec{C}$ on the line $\frac{1-x}{1}=\frac{y+4}{2}=\frac{z+2}{3}$, then $\abs{a+b+c}$ is equal to \rule{1cm}{0.15mm}.

\hfill{\brak{\text{Jan 2023}}}

\end{enumerate}
\end{document}

