%iffalse
\let\negmedspace\undefined
\let\negthickspace\undefined
\documentclass[journal,12pt,onecolumn]{IEEEtran}
\usepackage{cite}
\usepackage{amsmath,amssymb,amsfonts,amsthm}
\usepackage{algorithmic}
\usepackage{graphicx}
\usepackage{textcomp}
\usepackage{xcolor}
\usepackage{txfonts}
\usepackage{listings}
\usepackage{enumitem}
\usepackage{mathtools}
\usepackage{gensymb}
\usepackage{comment}
\usepackage[breaklinks=true]{hyperref}
\usepackage{tkz-euclide} 
\usepackage{listings}
\usepackage{gvv}                                        
%\def\inputGnumericTable{}                                 
\usepackage[latin1]{inputenc}     
\usepackage{xparse}
\usepackage{color}                                            
\usepackage{array}                                            
\usepackage{longtable}                                       
\usepackage{calc}                                             
\usepackage{multirow}
\usepackage{multicol}
\usepackage{hhline}                                           
\usepackage{ifthen}                                           
\usepackage{lscape}
\usepackage{tabularx}
\usepackage{array}
\usepackage{float}
\newtheorem{theorem}{Theorem}[section]
\newtheorem{problem}{Problem}
\newtheorem{proposition}{Proposition}[section]
\newtheorem{lemma}{Lemma}[section]
\newtheorem{corollary}[theorem]{Corollary}
\newtheorem{example}{Example}[section]
\newtheorem{definition}[problem]{Definition}
\newcommand{\BEQA}{\begin{eqnarray}}
\newcommand{\EEQA}{\end{eqnarray}}
\newcommand{\define}{\stackrel{\triangle}{=}}
\theoremstyle{remark}
\newtheorem{rem}{Remark}
% Marks the beginning of the document
\begin{document}
\title{Assignment 5}
\author{EE24Btech11024 - G. Abhimanyu Koushik}
\maketitle
\renewcommand{\thefigure}{\theenumi}
\renewcommand{\thetable}{\theenumi}
\subsection{Multiple Choice}
\begin{enumerate}

\item The relation $R=\cbrak{\brak{a,b}:\gcd\brak{a,b}=1,2a\neq b,a,b\in \mathbb{Z}}$ is:\rule{1cm}{0.15mm} 

\hfill{\brak{\text{Jan 2023}}}
\begin{enumerate}
\begin{multicols}{2}
\item Transitive but not reflexive
\item Symmetric but not transitive
\item Reflexive but not symmetric
\item Neither symmetric nor transitive
\end{multicols}
\end{enumerate}

\item The compound statement $\brak{\sim\brak{P\wedge Q}}\vee\brak{\brak{\sim P}\wedge Q}\implies\brak{\brak{\sim P}\wedge\brak{\sim Q}}$ is equivalent to

\hfill{\brak{\text{Jan 2023}}}
\begin{enumerate}
\begin{multicols}{2}
\item $\brak{\brak{\sim P}\vee Q}\wedge\brak{\brak{\sim Q}\vee P}$
\item $\brak{\sim Q}\vee P$
\item $\brak{\brak{\sim P}\vee Q}\wedge\brak{\sim Q}$
\item $\brak{\sim P}\vee Q$
\end{multicols}
\end{enumerate}

\item Let $f\brak{x} = \begin{cases} x^2\sin\brak{\frac{1}{x}} & x\neq 0, \\0 & x=0 .\end{cases}$; Then at $x=0$

\hfill{\brak{\text{Jan 2023}}}
\begin{enumerate}
\begin{multicols}{2}
\item $f$ is continuous but not differentiable
\item $f$ is continuous but $f^\prime$ is not continuous
\item $f$ and $f^\prime$ both are continuous 
\item $f^\prime$ is continuous but not differentiable
\end{multicols}
\end{enumerate}

\item The equation $x^2-4x+\sbrak{x}+3=x\sbrak{x}$, where $\sbrak{x}$ denotes greatest integer function, has:
\hfill{\brak{\text{Jan 2023}}}
\begin{enumerate}
\begin{multicols}{2}
\item Exactly two solutions in $\brak{-\infty,\infty}$
\item No solution
\item A unique solution in $\brak{-\infty,1}$
\item A unique solution in $\brak{-\infty,\infty}$
\end{multicols}
\end{enumerate}

\item Let $\Omega$ be the sample space and $A\subseteq \Omega$ be an event. Given below are two statements:\newline
$\brak{\text{S} 1}$: If $P\brak{A}=0$, then $A=\phi$\newline
$\brak{\text{S} 2}$: If $P\brak{A}=1$, then $A=\Omega$\newline
Then

\hfill{\brak{\text{Jan 2023}}}
\begin{enumerate}
\begin{multicols}{2}
\item Only $\brak{\text{S} 1}$ is true
\item Only $\brak{\text{S} 2}$ is true
\item Both $\brak{\text{S} 1}$ and $\brak{\text{S} 2}$ are true
\item Both $\brak{\text{S} 1}$ and $\brak{\text{S} 2}$ are false
\end{multicols}
\end{enumerate}
\end{enumerate}

\subsection{Numericals}
\begin{enumerate}

\item Let $C$ be the largest circle centred at $\brak{2,0}$ and inscribed in the ellipse $\frac{x^2}{36}+\frac{y^2}{16}=1$. If \brak{1,\alpha} lies on $C$, then $10\alpha^2$ is equal to \rule{1cm}{0.15mm}.

\hfill{\brak{\text{Jan 2023}}}

\item Suppose $\sum_{r=0}^{2023}r^2\times\comb{2023}{r} = 2023\times\alpha\times 2^{2022}$. Then the value of $\alpha$ is \rule{1cm}{0.15mm}.

\hfill{\brak{\text{Jan 2023}}}

\item The value of $12\int_{0}^{3}\abs{x^2-3x+2}dx$ is \rule{1cm}{0.15mm}.

\hfill{\brak{\text{Jan 2023}}}

\item The number of $9$ digit numbers, that can be formed using all the digits of the number $123412341$ so that the even digits occupy only even places is \rule{1cm}{0.15mm}.

\hfill{\brak{\text{Jan 2023}}}

\item Let $\lambda \in \mathbb{R}$ and let the equation $E$ be $\abs{x}^2-2\abs{x}+\abs{\lambda-3}=0$. Then the largest element in set $S=\cbrak{x+\lambda:x\text{ is an integer solution of }E}$ is \rule{1cm}{0.15mm}.

\hfill{\brak{\text{Jan 2023}}}

\item A boy needs to select $5$ courses from $12$ available courses, out of which $5$ courses are language courses. If  he can choose at most $2$ language courses, then the number of ways he can choose five courses is \rule{1cm}{0.15mm}.

\hfill{\brak{\text{Jan 2023}}}

\item Let a tangent to the curve $9x^2+16y^2=144$ intersect coordinate axes at points $\vec{A}$ and $\vec{B}$. Then, the minimum length of the line segment $AB$ is \rule{1cm}{0.15mm}.

\hfill{\brak{\text{Jan 2023}}}

\item The value of $\frac{8}{\pi}\int_{0}^{\frac{\pi}{2}}\frac{\brak{\cos x}^{2023}}{\brak{\sin x}^{2023}+\brak{\cos x}^{2023}}dx$ is \rule{1cm}{0.15mm}. 

\hfill{\brak{\text{Jan 2023}}}

\item The shortest distance between the lines $\frac{x-2}{3}=\frac{y+1}{2}=\frac{z-6}{2}$ and $\frac{x-6}{3}=\frac{1-y}{2}=\frac{z+8}{0}$ is equal to \rule{1cm}{0.15mm}.

\hfill{\brak{\text{Jan 2023}}}

\item The $4$\textsuperscript{th} term of GP is $500$ and its common ratio is $\frac{1}{m}$, $m\in \mathbb{N}$. Let $S_n$ denote the sum of the first $n$ terms of this GP. If $S_6>S_5+1$ and $S_7>S_6+\frac{1}{2}$, then the number of possible values of $m$ is \rule{1cm}{0.15mm}.

\hfill{\brak{\text{Jan 2023}}}

\end{enumerate}
\end{document}

