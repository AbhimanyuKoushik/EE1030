%iffalse
\let\negmedspace\undefined
\let\negthickspace\undefined
\documentclass[journal,12pt,onecolumn]{IEEEtran}
\usepackage{cite}
\usepackage{amsmath,amssymb,amsfonts,amsthm}
\usepackage{algorithmic}
\usepackage{graphicx}
\usepackage{textcomp}
\usepackage{xcolor}
\usepackage{txfonts}
\usepackage{listings}
\usepackage{enumitem}
\usepackage{mathtools}
\usepackage{gensymb}
\usepackage{comment}
\usepackage[breaklinks=true]{hyperref}
\usepackage{tkz-euclide} 
\usepackage{listings}
\usepackage{gvv}                                        
%\def\inputGnumericTable{}                                 
\usepackage[latin1]{inputenc}     
\usepackage{xparse}
\usepackage{color}                                            
\usepackage{array}                                            
\usepackage{longtable}                                       
\usepackage{calc}                                             
\usepackage{multirow}
\usepackage{multicol}
\usepackage{hhline}                                           
\usepackage{ifthen}                                           
\usepackage{lscape}
\usepackage{tabularx}
\usepackage{array}
\usepackage{float}
\newtheorem{theorem}{Theorem}[section]
\newtheorem{problem}{Problem}
\newtheorem{proposition}{Proposition}[section]
\newtheorem{lemma}{Lemma}[section]
\newtheorem{corollary}[theorem]{Corollary}
\newtheorem{example}{Example}[section]
\newtheorem{definition}[problem]{Definition}
\newcommand{\BEQA}{\begin{eqnarray}}
\newcommand{\EEQA}{\end{eqnarray}}
\newcommand{\define}{\stackrel{\triangle}{=}}
\theoremstyle{remark}
\newtheorem{rem}{Remark}
% Marks the beginning of the document
\begin{document}
\title{Assignment 2}
\author{EE24Btech11024 - G. Abhimanyu Koushik}
\maketitle
\renewcommand{\thefigure}{\theenumi}
\renewcommand{\thetable}{\theenumi}
\begin{enumerate}

\item The least value of $\abs{z}$ where $z$ is complex number which satisfies the inequality $\exp\brak{\frac{\brak{\abs{z}+3}\brak{\abs{z}-1}}{\abs{{\abs{z}+1}}}\log_e{2}}\ge\log_{\sqrt{2}}{\abs{5\sqrt{7}+9i}}$, $i=\sqrt{-1}$, is equal to

\hfill{\brak{\text{Mar 2021}}}
\begin{enumerate}
\begin{multicols}{4}
\item $8$
\item $3$
\item $\sqrt{5}$
\item $2$
\end{multicols}
\end{enumerate}

\item Let $f:S\rightarrow S$ where $S=\brak{0,\infty}$ be a twice differentiable function such that $f\brak{x+1}=xf\brak{x}$. If $g:S\rightarrow \mathbb{R}$ be defined as $g\brak{x} = \log_e{f\brak{x}}$, then the value of $\abs{g''\brak{5}-g''\brak{1}}$ is equal to 

\hfill{\brak{\text{Mar 2021}}}
\begin{enumerate}
\begin{multicols}{4}
\item $\frac{197}{144}$
\item $\frac{187}{144}$
\item $\frac{205}{144}$
\item $1$
\end{multicols}
\end{enumerate}

\item If $y=y\brak{x}$ is the solution of the differential equation $\brak{\frac{dy}{dx}}+\brak{\tan x}y=\sin x$, $0\leq x\leq \frac{\pi}{3}$, with $y\brak{0}=0$, then $y\brak{\frac{\pi}{4}}$ is equal to:
\hfill{\brak{\text{Mar 2021}}}
\begin{enumerate}
\begin{multicols}{4}
\item $\log_e{2}$
\item $\frac{1}{2}\log_e{2}$
\item $\frac{1}{2\sqrt{2}}\log_e{2}$
\item $\frac{1}{4}\log_e{2}$
\end{multicols}
\end{enumerate}

\item If the foot of perpendicular from the point \brak{4,2,8} on the $L_1:\frac{x-a}{l}=\frac{y-2}{3}=\frac{z-b}{4}$, $l\neq 0$ is \brak{3,5,7}, then find the shortest distance between the line $L_1$ and line $L_2:\frac{x-2}{3}=\frac{y-4}{4}=\frac{z-5}{5}$ is equal to

\hfill{\brak{\text{Mar 2021}}}
\begin{enumerate}
\begin{multicols}{4}
\item $\frac{\sqrt{2}}{3}$
\item $\frac{1}{\sqrt{3}}$
\item $\frac{1}{2}$
\item $\frac{1}{\sqrt{6}}$
\end{multicols}
\end{enumerate}

\item If \brak{x,y,z} be an arbitrary point lying on the plane $P$ which passes through the points \brak{42,0,0}, \brak{0,42,0}, and \brak{0,0,42}, then the value of expression $3+\frac{x-11}{\brak{y-19}^2\brak{z-12}^2}+\frac{y-19}{\brak{x-11}^2\brak{z-12}^2}+\frac{z-12}{\brak{x-11}^2\brak{y-19}^2} - \frac{x+y+z}{14\brak{x-11}\brak{y-19}\brak{z-12}}$ is equal to

\hfill{\brak{\text{Mar 2021}}}
\begin{enumerate}
\begin{multicols}{4}
\item $3$
\item $0$
\item $39$
\item $-45$
\end{multicols}
\end{enumerate}

\item Consider the integral $I=\int_{0}^{10}\frac{\sbrak{x}e^{\sbrak{x}}}{e^{x-1}}dx$, where $\sbrak{x}$ denotes the greatest integer less than or equal to $x$. Then the value of $I$ is equal to :

\hfill{\brak{\text{Mar 2021}}}
\begin{enumerate}
\begin{multicols}{4}
\item $45\brak{e-1}$
\item $45\brak{e+1}$
\item $9\brak{e-1}$
\item $9\brak{e+1}$
\end{multicols}
\end{enumerate}

\item Let $\vec{A}$\brak{-1, 1}, $\vec{B}$\brak{3, 4} and $\vec{C}$\brak{2, 0} be given three points. A line $y = mx$, $m > 0$, intersects lines $AC$ and $BC$ at point $\vec{P}$ and $\vec{Q}$ respectively. Let $A_1$ and $A_2$ be the areas of $\Delta ABC$ and $\Delta PQC$ respectively, such that $A_1 = 3A_2$, then the value of $m$ is equal to:

\hfill{\brak{\text{Mar 2021}}}
\begin{enumerate}
\begin{multicols}{4}
\item $\frac{4}{15}$
\item $1$
\item $2$
\item $3$
\end{multicols}
\end{enumerate}

\item Let $f$ be a real-valued function, defined on $\mathbb{R}-\cbrak{-1, 1}$ and given by $f\brak{x} = 3\log_e\brak{{\frac{\abs{\brak{x-1}}}{\abs{\brak{x+1}}}}}-\frac{2}{x-1}$. Then in which of the following intervals, function $f\brak{x}$ is increasing?

\hfill{\brak{\text{Mar 2021}}}
\begin{enumerate}
\begin{multicols}{2}
\item $\brak{-\infty,-1}\cup\brak{\lsbrak{\frac{1}{2}},\rbrak{\infty}-\cbrak{1}}$
\item $\lbrak{-1},\rsbrak{\frac{1}{2}}$
\item $\brak{-\infty,\infty}-\cbrak{-1,1}$
\item $\lbrak{-\infty},\rsbrak{\frac{1}{2}}-\cbrak{-1}$
\end{multicols}
\end{enumerate}

\item Let the lengths of intercepts on x-axis and y-axis made by the circle $x^2+y^2+ax+2ay+c=0$, $\brak{a<0}$ be $2\sqrt{2}$ and $2\sqrt{5}$, respectively. Then the shortest distance from origin to a tangent to this circle which is perpendicular to the line $x+2y=0$, is equal to:

\hfill{\brak{\text{Mar 2021}}}
\begin{enumerate}
\begin{multicols}{4}
\item $\sqrt{10}$
\item $\sqrt{6}$
\item $\sqrt{11}$
\item $\sqrt{7}$
\end{multicols}
\end{enumerate}

\item Let $A$ denote the event that a $6$-digit integer formed by $0$, $1$, $2$, $3$, $4$, $5$, $6$ without repetitions, be divisible by $3$. Then the probability of event $A$ is equal to:

\hfill{\brak{\text{Mar 2021}}}
\begin{enumerate}
\begin{multicols}{4}
\item $\frac{4}{9}$
\item $\frac{9}{56}$
\item $\frac{3}{7}$
\item $\frac{11}{27}$
\end{multicols}
\end{enumerate}

\item Let $\alpha \in \mathbb{R}$ be such that the function $f\brak{n} = \begin{cases} \frac{\cos^{-1}\brak{1-\cbrak{x}^2}\sin^{-1}\brak{1-\cbrak{x}}}{\cbrak{x}-\cbrak{x}^3} & x\neq 0, \\\alpha & x=0 .\end{cases}$ is continuous at $x=0$, where $\cbrak{x}=x-\sbrak{x}$, $\sbrak{x}$ is the greatest integer less than or equal to $x$. Then:

\hfill{\brak{\text{Mar 2021}}}
\begin{enumerate}
\begin{multicols}{4}
\item $\alpha=\frac{\pi}{4}$
\item No such $\alpha$ exists
\item $\alpha=0$
\item $\alpha=\frac{\pi}{\sqrt{2}}$
\end{multicols}
\end{enumerate}

\item The maximum value of $f\brak{x} = \mydet{\sin^{2}x &1+\cos^{2}x & \cos 2x\\1+\sin^{2}x & \cos^{2}x & b \cos 2x\\\sin^{2}x & \cos^{2}x & \sin 2x}$, $x \in \mathbb{R}$ is:

\hfill{\brak{\text{Mar 2021}}}
\begin{enumerate}
\begin{multicols}{4}
\item $\sqrt{7}$
\item $\sqrt{5}$
\item $5$
\item $\frac{3}{4}$
\end{multicols}
\end{enumerate}

\item Consider a rectangle $ABCD$ having $5$, $7$, $6$, $9$ points in the interior of the line segments $AB$, $BC$, $CD$, $DA$ respectively. Let $\alpha$ be the number of triangles having these points from the different sides as vertices and $\beta$ be the number of quadrilaterals having these points from different sides as vertices. Then $\brak{\beta-\alpha}$ is equal to: 

\hfill{\brak{\text{Mar 2021}}}
\begin{enumerate}
\begin{multicols}{4}
\item $1890$
\item $795$
\item $717$
\item $1173$
\end{multicols}
\end{enumerate}

\item Let $C$ be the locus of the mirror image of a point on the parabola $y^2=4x$ with respect to the line $y=x$. Then the equation of tangent to $C$ at $\vec{P}\brak{2,1}$ is:

\hfill{\brak{\text{Mar 2021}}}
\begin{enumerate}
\begin{multicols}{4}
\item $2x+y=5$
\item $x+2y=4$
\item $x+3y=5$
\item $x-y=1$
\end{multicols}
\end{enumerate}

\item Given that the inverse trigonometric functions take principal values only. Then, the number of real value of $x$ which satisfy $\sin^{-1}\brak{\frac{3x}{5}}+\sin^{-1}\brak{\frac{4x}{5}} = \sin^{-1}x$ is 

\hfill{\brak{\text{Mar 2021}}}
\begin{enumerate}
\begin{multicols}{4}
\item $1$
\item $2$
\item $3$
\item $0$
\end{multicols}
\end{enumerate}

\end{enumerate}
\end{document}

