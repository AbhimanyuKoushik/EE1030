%iffalse
\let\negmedspace\undefined
\let\negthickspace\undefined
\documentclass[journal,12pt,onecolumn]{IEEEtran}
\usepackage{cite}
\usepackage{amsmath,amssymb,amsfonts,amsthm}
\usepackage{algorithmic}
\usepackage{graphicx}
\usepackage{textcomp}
\usepackage{xcolor}
\usepackage{txfonts}
\usepackage{listings}
\usepackage{enumitem}
\usepackage{mathtools}
\usepackage{gensymb}
\usepackage{comment}
\usepackage[breaklinks=true]{hyperref}
\usepackage{tkz-euclide} 
\usepackage{listings}
\usepackage{gvv}                                        
%\def\inputGnumericTable{}                                 
\usepackage[latin1]{inputenc}     
\usepackage{xparse}
\usepackage{color}                                            
\usepackage{array}                                            
\usepackage{longtable}                                       
\usepackage{calc}                                             
\usepackage{multirow}
\usepackage{multicol}
\usepackage{hhline}                                           
\usepackage{ifthen}                                           
\usepackage{lscape}
\usepackage{tabularx}
\usepackage{array}
\usepackage{float}
\newtheorem{theorem}{Theorem}[section]
\newtheorem{problem}{Problem}
\newtheorem{proposition}{Proposition}[section]
\newtheorem{lemma}{Lemma}[section]
\newtheorem{corollary}[theorem]{Corollary}
\newtheorem{example}{Example}[section]
\newtheorem{definition}[problem]{Definition}
\newcommand{\BEQA}{\begin{eqnarray}}
\newcommand{\EEQA}{\end{eqnarray}}
\newcommand{\define}{\stackrel{\triangle}{=}}
\theoremstyle{remark}
\newtheorem{rem}{Remark}
% Marks the beginning of the document
\begin{document}
\title{Assignment 7}
\author{EE24Btech11024 - G. Abhimanyu Koushik}
\maketitle
\renewcommand{\thefigure}{\theenumi}
\renewcommand{\thetable}{\theenumi}
\subsection{Multiple Choice}
\begin{enumerate}

\item Let $\vec{A}\brak{-1,1}$ and $\vec{B}\brak{2,3}$ be two points and $\vec{P}$ be a variable point above the line $AB$ such that the area of $\Delta PAB$ is 10. If the locus of $\vec{P}$ is $ax+by=15$, then $5a+2b$ is:

\hfill{\brak{\text{Apr 2024}}}
\begin{enumerate}
\begin{multicols}{4}
\item $6$
\item $4$
\item $-\frac{12}{5}$
\item $-\frac{6}{5}$
\end{multicols}
\end{enumerate}

\item Let $\alpha\beta\neq 0$ and $A=\myvec{\beta & \alpha & 3\\\alpha & \alpha & \beta\\-\beta & \alpha & 2\beta}$. If $B=\myvec{3\alpha & -9 & 3\alpha\\-\alpha & 7 & -2\alpha \\ -2\alpha & 5 & -2\beta}$ is the matrix of cofactor elements of $A$, then $\det\brak{AB}$ is equal to:

\hfill{\brak{\text{Apr 2024}}}
\begin{enumerate}
\item $216$
\item $343$
\item $64$
\item $125$
\end{enumerate}

\item The value of $m$, $n$ for which the system of linear equations \newline $x+y+z=4$, \newline $2x+5y+5z=17$, \newline $x+2y+mz=n$ \newline has infinitely many solutions satisfy the equation:

\hfill{\brak{\text{Apr 2024}}}
\begin{enumerate}
\item $m^2+n^2-m-n=46$	
\item $m^2+n^2+mn=68$
\item $m^2+n^2+m+n=64$
\item $m^2+n^2-mn=39$
\end{enumerate}

\item Let $ABCD$ and $AEFG$ be squares of side $4$ and $2$ units respectively. The point $\vec{E}$ is on the line segment $AB$ and the point $\vec{F}$ is on the diagonal $AC$. Then the radius $r$ of the circle passing through the point $\vec{F}$ and touching the line segments $BC$ and $CD$ satisfies:
\begin{enumerate}
\begin{multicols}{4}
\item $r=1$
\item $r^2-8r+8=0$
\item $2r^2-8r+7=0$
\item $2r^2-4r+1=0$
\end{multicols}
\end{enumerate}

\item Let $\vec{a}=2\hat{i}+5\hat{j}-\hat{k}$, $\vec{b}=2\hat{i}-2\hat{j}+2\hat{k}$ and $\vec{c}$ be three vectors such that $\brak{\vec{c}+\hat{i}}\times\brak{\vec{a}+\vec{b}+\hat{i}}=\vec{a}\times\brak{\vec{c}+\hat{i}}$. If $\vec{a}\cdot\vec{c}=-29$, then $\vec{c}\cdot\brak{-2\hat{i}+\hat{j}+\hat{k}}$ is equal to:

\hfill{\brak{\text{Apr 2024}}}
\begin{enumerate}
\begin{multicols}{4}
\item $15$
\item $12$
\item $5$
\item $10$
\end{multicols}
\end{enumerate}
\end{enumerate}

\subsection{Numericals}
\begin{enumerate}

\item Let the maximum and minimum values of $\brak{\sqrt{8x-x^2-12}-4}^2+\brak{x-7}^2$, $x\in\mathbb{R}$ be $M$ and $m$, respectively. Then $M^2-m^2$ is equal to \rule{1cm}{0.15mm}.

\hfill{\brak{\text{Apr 2024}}}

\item Let the point \brak{-1,\alpha,\beta} lie on the line of the shortest distance between the lines $\frac{x+2}{-3}=\frac{y-2}{4}=\frac{z-5}{2}$ and $\frac{x+2}{-1}=\frac{y+6}{2}=\frac{z-1}{0}$. Then $\brak{\alpha=\beta}^2$ is equal to \rule{1cm}{0.15mm}.

\hfill{\brak{\text{Apr 2024}}}

\item The number of real solutions of the equation $x\abs{x+5}+2\abs{x+7}-2=0$ is \rule{1cm}{0.15mm}.

\hfill{\brak{\text{Apr 2024}}}

\item Let $y=y\brak{x}$ be the solution to the differential equation $\frac{dy}{dx}+\frac{2x}{\brak{1+x^2}^2}y=xe^{\frac{1}{1+x^2}};y\brak{0}=0$. Then the area enclosed by the curve $f\brak{x}=y\brak{x}e^{-\frac{1}{1+x^2}}$ and the line $y-x=4$ is \rule{1cm}{0.15mm}.

\hfill{\brak{\text{Apr 2024}}}

\item Let a line perpendicular to the line $2x-y=10$ touch the parabola $y^2=4\brak{x-9}$ at the point $\vec{P}$. The distance of the point $\vec{P}$ from the centre of the circle $x^2+y^2-14x-8y+56=0$ is \rule{1cm}{0.15mm}.

\hfill{\brak{\text{Apr 2024}}}

\item The number of solutions of $\sin^{2}x+\brak{2+2x-x^2}\sin x - 3\brak{x-1}^2=0$, where $-\pi\leq x\leq \pi$, is \rule{1cm}{0.15mm}.

\hfill{\brak{\text{Apr 2024}}}

\item Let the mean and the standard deviation of a probability distribution\\
\begin{center}
   \begin{tabular}{| p{6.5cm} | p{5cm} |}
\hline
\textbf{Column I} & \textbf{Column II} \\
\hline
P. Exfoliated silicates filled butyl rubber & 1. Automobile pistons \\
\hline
Q. Fibre reinforced aluminium alloy & 2. Contact lenses \\
\hline
R. Silicon carbide whiskers reinforced alumina & 3. Ski boards \\
\hline
S. Carbon particles reinforced plastic composites & 4. Tennis balls \\
\hline
& 5. Cutting tool inserts for machining \\
\hline
\end{tabular}

\end{center}
be $\mu$ and $\sigma$, respectively. Then $\sigma+\mu$ is equal to \rule{1cm}{0.15mm}.

\hfill{\brak{\text{Apr 2024}}}

\item If $1+\frac{\sqrt{3}-\sqrt{2}}{2\sqrt{3}}+\frac{5-2\sqrt{6}}{18}+\frac{9\sqrt{3}-11\sqrt{2}}{36\sqrt{3}}+\frac{49-20\sqrt{6}}{180}+\dots$ upto $\infty=2+\brak{\sqrt{\frac{b}{a}}+1}\log_e{\frac{a}{b}}$, where $a$ and $b$ are integers with $\gcd\brak{a,b}=1$, then $11a+18b$ is equal to \rule{1cm}{0.15mm}. 

\hfill{\brak{\text{Apr 2024}}}

\item If $f\brak{t}=\int_{0}^{\pi}\frac{2xdx}{1-\cos^{2}t\sin^{2}x}$, $0<t<\pi$, then the value of $\int_{0}^{\frac{\pi}{2}}\frac{\pi^{2}dt}{f\brak{t}}$ equals \rule{1cm}{0.15mm}.

\hfill{\brak{\text{Apr 2024}}}

\item Let $a>0$ be a root of the equation $2x^2+x-2=0$. If $\lim_{x\to\frac{1}{a}}\frac{16\brak{1-\cos\brak{2+x-2x^2}}}{\brak{1-ax}^2}=\alpha+\beta\sqrt{17}$, where $\alpha,\beta\in\mathbb{Z}$, then $\alpha+\beta$ is equal to \rule{1cm}{0.15mm}.

\hfill{\brak{\text{Apr 2024}}}

\end{enumerate}
\end{document}

