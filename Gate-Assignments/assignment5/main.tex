%iffalse
\let\negmedspace\undefined
\let\negthickspace\undefined
\documentclass[journal,12pt,onecolumn]{IEEEtran}
\usepackage{cite}
\usepackage{amsmath,amssymb,amsfonts,amsthm}
\usepackage{algorithmic}
\usepackage{graphicx}
\usepackage{textcomp}
\usepackage{xcolor}
\usepackage{txfonts}
\usepackage{listings}
\usepackage{enumitem}
\usepackage{mathtools}
\usepackage{gensymb}
\usepackage{comment}
\usepackage[breaklinks=true]{hyperref}
\usepackage{tkz-euclide} 
\usepackage{listings}
\usepackage{gvv}                                        
%\def\inputGnumericTable{}                                 
\usepackage[latin1]{inputenc}     
\usepackage{xparse}
\usepackage{color}                                            
\usepackage{array}                                            
\usepackage{longtable}                                       
\usepackage{calc}                                             
\usepackage{multirow}
\usepackage{multicol}
\usepackage{hhline}                                           
\usepackage{ifthen}                                           
\usepackage{lscape}
\usepackage{tabularx}
\usepackage{array}
\usepackage{float}
\newtheorem{theorem}{Theorem}[section]
\newtheorem{problem}{Problem}
\newtheorem{proposition}{Proposition}[section]
\newtheorem{lemma}{Lemma}[section]
\newtheorem{corollary}[theorem]{Corollary}
\newtheorem{example}{Example}[section]
\newtheorem{definition}[problem]{Definition}
\newcommand{\BEQA}{\begin{eqnarray}}
\newcommand{\EEQA}{\end{eqnarray}}
\newcommand{\define}{\stackrel{\triangle}{=}}
\theoremstyle{remark}
\newtheorem{rem}{Remark}
% Marks the beginning of the document
\begin{document}
\title{Assignment 5}
\author{EE24Btech11024 - G. Abhimanyu Koushik}
\maketitle
\renewcommand{\thefigure}{\theenumi}
\renewcommand{\thetable}{\theenumi}
\begin{enumerate}

\item Match the techniques listed in Column $I$ with the characteristics of the materials measured in Column $II$.
\\\begin{table}[h!]    
  \centering
  \begin{tabular}{| p{6.5cm} | p{5cm} |}
\hline
\textbf{Column I} & \textbf{Column II} \\
\hline
P. Exfoliated silicates filled butyl rubber & 1. Automobile pistons \\
\hline
Q. Fibre reinforced aluminium alloy & 2. Contact lenses \\
\hline
R. Silicon carbide whiskers reinforced alumina & 3. Ski boards \\
\hline
S. Carbon particles reinforced plastic composites & 4. Tennis balls \\
\hline
& 5. Cutting tool inserts for machining \\
\hline
\end{tabular}

\end{table}\\

\hfill{\brak{\text{XE 2015}}}
\begin{enumerate}
\begin{multicols}{2}
\item P-$2$, Q-$3$, R-$4$, S-$1$
\item P-$5$, Q-$4$, R-$5$, S-$1$
\item P-$2$, Q-$4$, R-$1$, S-$3$
\item P-$3$, Q-$5$, R-$4$, S-$2$
\end{multicols}
\end{enumerate}

\item The mass of an electron would increase \rule{1cm}{0.15mm} times its original mass if it travels at $96\%$ of the speed of light.

\hfill{\brak{\text{XE 2015}}}

\item With increasing temperature from $15^\degree C$ in winter to $40^\degree C$ in summer, the length of an iron rail track increases by $0.05$ $cm$. Calculate the original length of the iron rail track in cm. \\\brak{\text{linear thermal expansion coefficient of iron is } 11.0 \times 10^{-6} K^{-1}}

\hfill{\brak{\text{XE 2015}}}

\item What is the thickness \brak{\text{in }\mu m}of a germanium crystal layer that would be required for absorbing $80\%$ of the incident radiation whose wavelength is $1.3 \mu m$? The absorption coefficient \brak{\alpha} of germanium at $1.3 \mu m$ is $3.3\times 10^5 m^{-1}$

\hfill{\brak{\text{XE 2015}}}

\item A $1$ $kg$ sacrificial anode of $Mg$ \brak{\text{atomic weight: $24.31$ $amu$}} is attached to the base of a ship. If the anode lasts for $60$ days, what is the average corrosion current \brak{\text{in Amperes}} during that period?

\hfill{\brak{\text{XE 2015}}}

\item A capacitor has a $0.075$ $cm$ thick $BaTiO_3$ dielectric with a dielectric constant of $2000$ and an electrode area of $0.2$ $cm^2$. What is the capacitance of this capacitor in $nF$?

\hfill{\brak{\text{XE 2015}}}

\item A hot pressed ceramic composite material consists of $30$ volume $\%$ $SiC$ whiskers in an $Al_{2}O_{3}$ matrix. The measured buld density of this composite is $3.65$ $g$ $cm^{-3}$, estimate the porosity \brak{\%} of the composite, assuming that the linear rule of mixtures is valid in this case.

\hfill{\brak{\text{XE 2015}}}

\item Match the technical ceramics listed in Column $I$ with their common applications listed in Column $II$
\\\begin{table}[h!]    
  \centering
  \begin{tabular}{| p{4cm} | p{6cm} |}
    \hline
    \textbf{Column I} & \textbf{Column II} \\
    \hline
    P. Y-doped $ZrO_2$ & 1. Lasers \\
    \hline
    Q. $UO_2$ & 2. Turbine Engine \\
    \hline
    R. $Si_{3}N_{4}$ & 3. Integrated circuit substrate \\
    \hline
    S. $AlN$ & 4. Oxygen sensor \\
    \hline
    T. $Cr$ doped $Al_{2}O_{3}$ & 5. Nuclear fuel \\
    \hline
    & 6. Thermistor \\
    \hline
\end{tabular}

\end{table}\\

\hfill{\brak{\text{XE 2015}}}
\begin{enumerate}
\begin{multicols}{2}
\item P-$6$, Q-$4$, R-$5$, S-$1$, T-$3$
\item P-$4$, Q-$5$, R-$2$, S-$3$, T-$1$
\item P-$3$, Q-$1$, R-$2$, S-$6$, T-$5$
\item P-$1$, Q-$4$, R-$5$, S-$2$, T-$1$
\end{multicols}
\end{enumerate}

\item Creep in metals is defined as

\hfill{\brak{\text{XE 2015}}}
\begin{enumerate}
\item the maximum energy a solid can absorb elastically
\item the maximum energy a solid can absorb by plastic deformation
\item the stress at which plastic deformation starts
\item slow plastic deformation due to diffusion of atoms usually at high temperatures \brak{T>\text{half the melting point}}
\end{enumerate}

\item Calculate the planar density of the \brak{100} plane in an fcc crystal given that $R$ is the atomic radius of the element.

\hfill{\brak{\text{XE 2015}}}
\begin{enumerate}
\begin{multicols}{4}
\item $0.25R^2$
\item $\frac{0.25}{R^2}$
\item $\frac{1}{R^2}$
\item $\frac{4}{R^2}$
\end{multicols}
\end{enumerate}

\item The diffusion coefficient of copper atoms in aluminium if found to be $1.28\times 10^{-22}$ $m^2s^{-1}$ at $T=400$ $K$ and $5.75\times 10^{-19}$ $m^2s^{-1}$ at $T=500$ $K$. Find the temperature \brak{\text{in Kelvin}} at which the value of the diffusion coefficient is $10^{-16}$ $m^2s^-1$

\hfill{\brak{\text{XE 2015}}}

\item Calculate the density of copper in $kgm^{-3}$ given that copper has an fcc lattice with a lattice parameter of $0.365$ $nm$. Copper has an atomic weight of $63.54$ $amu$.

\hfill{\brak{\text{XE 2015}}}

\item What would be the maximum number of electron-hole pairs that can be generated using a silicon detector irradiated by X-ray of energy $1.54$ $keV$. The band gap of silicon is $1.1$ $eV$.

\hfill{\brak{\text{XE 2015}}}

\end{enumerate}
\end{document}

