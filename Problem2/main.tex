\let\negmedspace\undefined
\let\negthickspace\undefined
\documentclass[journal]{IEEEtran}
\usepackage[a5paper, margin=10mm, onecolumn]{geometry}
%\usepackage{lmodern} % Ensure lmodern is loaded for pdflatex
\usepackage{tfrupee} % Include tfrupee package

\setlength{\headheight}{1cm} % Set the height of the header box
\setlength{\headsep}{0mm}     % Set the distance between the header box and the top of the text

\usepackage{gvv-book}
\usepackage{gvv}
\usepackage{cite}
\usepackage{amsmath,amssymb,amsfonts,amsthm}
\usepackage{algorithmic}
\usepackage{graphicx}
\usepackage{textcomp}
\usepackage{xcolor}
\usepackage{txfonts}
\usepackage{listings}
\usepackage{enumitem}
\usepackage{mathtools}
\usepackage{gensymb}
\usepackage{comment}
\usepackage[breaklinks=true]{hyperref}
\usepackage{tkz-euclide} 
\usepackage{listings}
% \usepackage{gvv}                                        
\def\inputGnumericTable{}                                 
\usepackage[latin1]{inputenc}                                
\usepackage{color}                                            
\usepackage{array}                                            
\usepackage{longtable}                                       
\usepackage{calc}                                             
\usepackage{multirow}                                         
\usepackage{hhline}                                           
\usepackage{ifthen}                                           
\usepackage{lscape}
\begin{document}

\bibliographystyle{IEEEtran}
\vspace{3cm}

\title{1.1.5.23}
\author{EE24BTECH11024 - G. Abhimanyu Koushik
}
% \maketitle
% \newpage
% \bigskip
{\let\newpage\relax\maketitle}

\renewcommand{\thefigure}{\theenumi}
\renewcommand{\thetable}{\theenumi}
\setlength{\intextsep}{10pt} % Space between text and floats


\numberwithin{equation}{enumi}
\numberwithin{figure}{enumi}
\renewcommand{\thetable}{\theenumi}


\textbf{Question}:\\
Show that the points $\vec{A}$\brak{-2\hat{i}+3\hat{j}+5\hat{k}}, $\vec{B}$\brak{\hat{i}+2\hat{j}+3\hat{k}} and $\vec{C}$\brak{7\hat{i}-\hat{k}} are collinear.
\\
\textbf{Solution: }
%\begin{table}[h!]    
 % \centering
  %\begin{tabular}[12pt]{ |c|c|}
    \hline
    \textbf{Symbol} & \textbf{Description} \\
    \hline
    \textbf{$a$} & length of side BC\\
    \hline
    \textbf{$b$} & length of side CA\\
    \hline
    \textbf{$c$} & length of side AB\\
    \hline
    $\angle A$ & angle at vertex A\\
    \hline
    $\angle B$ & angle at vertex B\\
    \hline
    $\angle C$ & angle at vertex C\\
    \hline
    \textbf{$K$} & Perimeter of triangle\\
    \hline
    \textbf{$x$} & $\frac{a}{K}$\\
    \hline
    \textbf{$y$} & $\frac{b}{K}$\\
    \hline
    \textbf{$z$} & $\frac{c}{K}$\\
    \hline
    \end{tabular}

  %\caption{Variables Used}
  %\label{tab10.5.3.9.1}
%\end{table}
The Collinearity matrix is given by\\
\begin{align}
\myvec{
   \vec{B}-\vec{A} & \vec{C}-\vec{A}
 }^T = \myvec{
   3 & -1 & -2
   \\
   9 & -3 & -6
   }\\
 \xleftrightarrow[]{R_2 \leftarrow {R_1-3R_2}}
 \myvec{
   3 & -1 & -2
   \\
   0 & 0 & 0
   }
\end{align}
Since the rank of the Collinearity matrix is $1$, the points are collinear
\begin{figure}[h!]
   \centering
   \includegraphics[width=0.7\linewidth]{figs/fig.pdf}
   \caption{Line through the given points}
   \label{stemplot}
\end{figure}
\end{document}  


